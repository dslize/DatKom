\documentclass{../exercisesheet}

\title{Datenkommunikation und Informationssysteme, Übung 8}
\author{
    Domenic Quirl \\ 354437
    \and
    Julian Schakib \\ 353889
    \and 
    Daniel Schleiz \\ 356092
}

\renewcommand{\Exercise}{Aufgabe}
\date{Übungsgruppe 14}

\usepackage{float}
%\usepackage{siunitx}
\usepackage{color}
\usepackage{multirow}
\usepackage{float}

\begin{document}
\maketitle
\pointtable


\begin{exercise}{6}
\begin{subexercise}

\end{subexercise}
\begin{subexercise}
\begin{description}
\item[i)] \ \\

\item[ii)] \ \\

\end{description}
\end{subexercise}
\begin{subexercise}

\end{subexercise}
\end{exercise}


\begin{exercise}{4}
\begin{subexercise}
\begin{description}
\item[i)] \ \\

\item[ii)] \ \\

\end{description}
\end{subexercise}
\begin{subexercise}

\end{subexercise}
\end{exercise}


\begin{exercise}{5}
\begin{subexercise}
Da $p=13$ und $q=23$, ist $n=p\cdot q = 299$. Der public key ist also $\langle 61,299 \rangle$. Zudem ist $\Phi(299)=(13-1)\cdot(23-1)= 264$. Finde nun $d$ so, dass
$d \cdot e = d \cdot 61 \equiv_{264} 1$. Verwende den erweiterten Algorithmus von \textsc{Euklid}:
\begin{align*}
264 &= 4 \cdot 61 + 20	&	1&=61-3\cdot 20 \\
61 &= 3 \cdot 20 + 1	&	 &= 61-3 \cdot(264-4\cdot 61) \\
20 &= 20 \cdot 1 + 0	&	 &= -3 \cdot 264 + 13 \cdot 61 \\
\null &				&	 &\equiv_{264} 13 \cdot 61
\end{align*}
Nun folgt also, dass der private key $\langle 13, 299 \rangle$ ist.
\begin{itemize}
\item Verschlüssele $m_1=21$: $c_1 = 21^{61} \equiv_{299} 281$.
\item Entschlüssele $c_2=291$: $m_2=291^{13} \equiv_{299} 5$.
\end{itemize}
\end{subexercise}
\begin{subexercise}
Es ist bekannt, dass $n=91$. Finde durch geschicktes Ausprobieren heraus, dass die Primfaktorzerlegung von $n$ gegeben ist durch $p=7$, $q=13$, da $91 = 7 \cdot 13$.
Außerdem ist $\Phi(n)=6 \cdot 12 = 72$. Suche nun $d$, sodass $d \cdot e= d \cdot 29 \equiv_{72} 1$. Verwende erneut den erweiterten Algorithmus von \textsc{Euklid}:
\begin{align*}
72 &= 2 \cdot 29 + 14	&	1 &= 29 - 2 \cdot 14 \\
29 &= 2 \cdot 14 + 1	&	  &= 29 - 2 \cdot (72-2 \cdot 29) \\
14 &= 14 \cdot 1 + 0	&	  &= -2 \cdot 72 + 5 \cdot 29 \\
\null &				&	  &\equiv_{72} 5 \cdot 29
\end{align*}
Es folgt, dass der private key gegeben ist durch $\langle 5,91\rangle$.
\begin{itemize}
\item Dekodiere $c=3$ zu $m=3^5 \equiv_{91} 61$.
\end{itemize}
\end{subexercise}
\end{exercise}



\end{document}

