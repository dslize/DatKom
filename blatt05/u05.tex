\documentclass{../exercisesheet}

\title{Datenkommunikation und Informationssysteme, Übung 5}
\author{
    Domenic Quirl \\ 354437
    \and
    Julian Schakib \\ 353889
    \and 
    Daniel Schleiz \\ 356092
}

\renewcommand{\Exercise}{Aufgabe}
\date{Übungsgruppe 14}

\usepackage{float}
%\usepackage{siunitx}
\usepackage{color}
\usepackage{multirow}

\begin{document}
\maketitle
\pointtable


\begin{exercise}{4}
\begin{subexercise}

\end{subexercise}
\begin{subexercise}

\end{subexercise}
\begin{subexercise}

\end{subexercise}
\end{exercise}


\begin{exercise}{5}
\begin{subexercise}
Berechne zunächst die Latenzen (Länge geteilt durch die Ausbreitungsgeschwindigkeit) und die maximalen Datenraten zwischen den Zwischenknoten:\\
\begin{table}[h]
\centering
\begin{tabular}{c|c|c}
 & Latenz & max. Datenrate \\ \hline
$S\rightarrow R_1$      & 2,5$\mu$s & 1 Mbit/s \\ \hline
$R_1 \rightarrow R_2$ & 25$\mu$s & 1000 Mbit/s \\ \hline
$R_2 \rightarrow D$     & 5$\mu$s & 10 Mbit/s
\end{tabular}
\end{table}\\
(Bei NRZ wird pro Schritt ein Bit kodiert, also in dem Fall  entspricht 1 MBaud gerade 1 Mbit/s. Bei 4B/5B werden 4 Bits in 5 Schritten übertragen, d.h. $1250 \cdot 0,8$ Mbit/s. Für den
Manchester Leitungscode werden zwei Schritte benötigt, um ein Bit zu übertragen, also $20 \cdot 0,5$ Mbit/s.)\\ \ \\
\textbf{(i)}
\begin{adjustwidth}{2em}{0em}\vspace{-\baselineskip}
Für $P=75 \cdot 8=600$ Bit benötigt das Paket (inklusive Header von 160 Bit)
\[
\frac{760 \text{Bit}}{10^6 \text{Bit/s}} + \frac{760 \text{Bit}}{1000 \cdot 10^6 \text{Bit/s}} + \frac{760 \text{Bit}}{10 \cdot 10^6 \text{Bit/s}} + 32,5 \cdot 10^{-6}\text{s} + 
2 \cdot 10^{-6}\text{s} = 0,87126 \cdot 10^{-3}\text{s}
\]
(Benötigte Zeit zur Übertragung der jeweiligen Leitungen plus die summierten Latenzen plus die Verarbeitungszeiten der Zwischenstationen $R_i$.)
\end{adjustwidth}
\textbf{(ii)}
\begin{adjustwidth}{2em}{0em}\vspace{-\baselineskip}
Für $P=1500 \cdot 8=12000$ Bit benötigt das Paket (inklusive Header von 160 Bit)
\[
\frac{12160 \text{Bit}}{10^6 \text{Bit/s}} + \frac{12160 \text{Bit}}{1000 \cdot 10^6 \text{Bit/s}} + \frac{12160 \text{Bit}}{10 \cdot 10^6 \text{Bit/s}} + 32,5 \cdot 10^{-6}\text{s} + 
2 \cdot 10^{-6}\text{s} = 13,42266 \cdot 10^{-3}\text{s}
\]
\end{adjustwidth}
\textbf{(iii)}
\begin{adjustwidth}{2em}{0em}\vspace{-\baselineskip}
Für $P=30000 \cdot 8=240000$ Bit benötigt das Paket (inklusive Header von 160 Bit)
\[
\frac{240160 \text{Bit}}{10^6 \text{Bit/s}} + \frac{240160 \text{Bit}}{1000 \cdot 10^6 \text{Bit/s}} + \frac{240160 \text{Bit}}{10 \cdot 10^6 \text{Bit/s}} + 32,5 \cdot 10^{-6}\text{s}
+ 2 \cdot 10^{-6}\text{s} = 264,45066 \cdot 10^{-3}\text{s}
\] \ \\
\end{adjustwidth}
\end{subexercise}
\begin{subexercise}
\textbf{(i)}
\begin{adjustwidth}{2em}{0em}\vspace{-\baselineskip}
Die Nachricht wird in $\frac{30000}{75}=400$ Paketen verschickt und die Versendung benötigt demnach $400 \cdot 0,87126 \cdot 10^{-3}\text{s}=348,504$ms.
\end{adjustwidth}
\textbf{(ii)}
\begin{adjustwidth}{2em}{0em}\vspace{-\baselineskip}
Die Nachricht wird in $\frac{30000}{1500}=20$ Paketen verschickt und die Versendung benötigt demnach $20 \cdot 13,42266 \cdot 10^{-3}\text{s}=268,4532$ms.
\end{adjustwidth}
\textbf{(iii)}
\begin{adjustwidth}{2em}{0em}\vspace{-\baselineskip}
Die Nachricht wird in einem Paket verschickt und die Versendung benötigt demnach $264,45066$ms.
\end{adjustwidth}
\end{subexercise}
\end{exercise}


\begin{exercise}{4}
Damit die Adressen nicht zu lang werden, wird im Folgenden, wenn die Betrachtung der Binärdarstellung eines gewissen Teils notwendig ist, nur der relevante Teil binär dargestellt. \\
\begin{subexercise}
Der IP-Adressbereich \texttt{137.226.40.0/21} impliziert eine Subnetzmaske mit 21 Einsen, d.h. die Subnetzmaske \texttt{255.255.248.0 = 255.255.11111000.0} (und \texttt{137.226.40.0 = 137.226.00101000.0}).
\begin{itemize}
\item Verundet man IP 1 mit der Subnetzmaske, so erhält man die Adresse \texttt{137.226.48.0 = \\ 137.226.00110000.0}. Die Adresse liegt also nicht im gegebenen Adressbereich, da 
	\texttt{00110 $\neq$ 00101}.
\item Man sieht direkt, dass IP 2 nicht im Adressbereich liegt, weil schon im ersten 8 Bit Teil der Adresse ein Unterschied vorliegt und dieser Teil offensichtlich bei der Subnetzmaske verundet
	wird. (\texttt{136 $\neq$ 137}).
\end{itemize}
\end{subexercise}
\begin{subexercise}
\begin{itemize}
\item Um 900 Rechner in LAN 1 zu adressieren, benötigt man 10 Bit ($2^9 - 2 = 510 < 900 < 1022 = 2^{10}-2$). Somit kriegt LAN 1 den Adressbereich \texttt{137.226.40.0/22} mit
	Subnetzmaske \texttt{255.255.252.0}. (So klein wie möglich, da 11 Bit zur Verfügung standen.) Das Subnetz erhält als Netz-ID die niedrigste Adresse des Subnetzes, also 
	\texttt{137.226.40.0}. (Hosts: \texttt{137.226.001010xx.xxxxxxxx})
	\\ Der restliche Adressbereich umfasst \texttt{137.226.44.0/22}.
\item Um 200 Rechner in LAN 2 zu adressieren, benötigt man 8 Bit ($2^7 - 2 = 126 < 200 < 254 = 2^8-2$). Der kleinstmögliche Adressbereich für LAN 2 wäre dann 
	\texttt{137.226.44.0/24} mit Subnetzmaske \texttt{255.255.255.0} und Netz-ID \texttt{137.226.44.0}. (Hosts: \texttt{137.226.00101100.xxxxxxxx})
\item Um 500 Rechner in LAN 3 zu adressieren, benötigt man 9 Bit ($2^8 - 2 = 254 < 500 < 510 = 2^9-2$). Der kleinstmögliche Adressbereich für LAN 3 wäre dann 
	\texttt{137.226.46.0/23} mit Subnetzmaske \texttt{255.255.254.0} und Netz-ID \texttt{137.226.46.0}. (Hosts: \texttt{137.226.0010101x.xxxxxxxx})
\item Um 75 Rechner adressieren zu können benötigt man 7 Bit. Der kleinst mögliche Adressbereich für LAN 4 wäre dann \texttt{137.226.45.0/25} mit Subnetzmaske 
	\texttt{255.255.255.128} und Netz-ID \texttt{137.226.45.0}. (Hosts: \texttt{137.226.00101001.0xxxxxxx}).
\end{itemize}
\ \\Nach der Einteilung ist noch der Adressbereich \texttt{137.226.45.128/25} frei.
\end{subexercise}
\begin{subexercise}
Die höchste Adresse eines Subnetzes ist für Broadcast reserviert, weshalb diese nicht vergeben wird. Nach den Vergaberegeln der Aufgabenstellung ergebt sich folgende Verteilung von
IP-Adressen:
\begin{itemize}
\item In LAN 1 erhält A.if1 \texttt{137.226.40.1}, h1 kriegt \texttt{137.226.43.254} und h2 kriegt \texttt{137.226.43.253}
\item A.if2: \texttt{137.226.44.1}, B.if1: \texttt{137.226.44.2}, h3: \texttt{137.226.44.254}
\item B.if2: \texttt{137.226.46.1}, h4: \texttt{137.226.47.254}
\item B.if3: \texttt{137.226.45.1}, h5: \texttt{137.226.45.126}
\end{itemize}
\end{subexercise}
\end{exercise}


\begin{exercise}{2}
\begin{subexercise}
\begin{center}
\begin{tabular}{c|cc|cc|cc}
\multirow{2}{*}{Protokoll} & \multicolumn{2}{c}{lokal} & \multicolumn{2}{c}{global} & \multicolumn{2}{c}{Ziel} \\
 & IP-Adresse & Port & IP-Adresse & Port & IP-Adresse & Port \\
\hline
TCP & 10.0.0.1 & 8051 & 137.226.12.228 & 8051 & 137.226.13.142 & 443 \\
UDP & 10.0.0.3 & 4711 & 137.226.12.228 & 4711 & 8.8.8.8 & 53 \\
UDP & 10.0.0.4 & 4711 & 137.226.12.228 & 4712 & 8.8.8.8 & 53 \\
\end{tabular}
\end{center}
\end{subexercise}
\begin{subexercise}
Die Tabelle müsste um einen Eintrag ergänzt werden, welcher eingehende Anfragen auf Port 80 an Port 8888 des Rechners B weiterleitet, also ein Eintrag der Form\\
\begin{center}
\begin{tabular}{c|cc|cc|cc}
\multirow{2}{*}{Protokoll} & \multicolumn{2}{c}{lokal} & \multicolumn{2}{c}{global} & \multicolumn{2}{c}{Ziel} \\
 & IP-Adresse & Port & IP-Adresse & Port & IP-Adresse & Port \\
\hline
TCP & 10.0.0.2 & 8888 & 137.226.12.228 & 80 & - & - \\
\end{tabular}
\end{center}
\end{subexercise}
\end{exercise}

\end{document}
