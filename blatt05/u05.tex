\documentclass{../exercisesheet}

\title{Datenkommunikation und Informationssysteme, Übung 5}
\author{
    Domenic Quirl \\ 354437
    \and
    Julian Schakib \\ 353889
    \and 
    Daniel Schleiz \\ 356092
}

\renewcommand{\Exercise}{Aufgabe}
\date{Übungsgruppe 14}

\usepackage{float}
%\usepackage{siunitx}
\usepackage{color}
\usepackage{multirow}

\begin{document}
\maketitle
\pointtable


\begin{exercise}{4}
\begin{subexercise}

\end{subexercise}
\begin{subexercise}

\end{subexercise}
\begin{subexercise}

\end{subexercise}
\end{exercise}


\begin{exercise}{5}
\begin{subexercise}
Berechne zunächst die Latenzen (Länge geteilt durch die Ausbreitungsgeschwindigkeit) und die maximalen Datenraten zwischen den Zwischenknoten:\\
\begin{table}[h]
\centering
\begin{tabular}{c|c|c}
 & Latenz & max. Datenrate \\ \hline
$S\rightarrow R_1$      & 2,5$\mu$s & 1 Mbit/s \\ \hline
$R_1 \rightarrow R_2$ & 25$\mu$s & 1000 Mbit/s \\ \hline
$R_2 \rightarrow D$     & 5$\mu$s & 10 Mbit/s
\end{tabular}
\end{table}\\
(Bei NRZ wird pro Schritt ein Bit kodiert, also in dem Fall  entspricht 1 MBaud gerade 1 Mbit/s. Bei 4B/5B werden 4 Bits in 5 Schritten übertragen, d.h. $1250 \cdot 0,8$ Mbit/s. Für den
Manchester Leitungscode werden zwei Schritte benötigt, um ein Bit zu übertragen, also $20 \cdot 0,5$ Mbit/s.)\\ \ \\
\textbf{(i)}
\begin{adjustwidth}{2em}{0em}\vspace{-\baselineskip}
Für $P=75 \cdot 8=600$ Bit benötigt das Paket (inklusive Header von 160 Bit)
\[
\frac{760 \text{Bit}}{10^6 \text{Bit/s}} + \frac{760 \text{Bit}}{1000 \cdot 10^6 \text{Bit/s}} + \frac{760 \text{Bit}}{10 \cdot 10^6 \text{Bit/s}} + 32,5 \cdot 10^{-6}\text{s} + 
2 \cdot 10^{-6}\text{s} = 0,87126 \cdot 10^{-3}\text{s}
\]
(Benötigte Zeit zur Übertragung der jeweiligen Leitungen plus die summierten Latenzen plus die Verarbeitungszeiten der Zwischenstationen $R_i$.)
\end{adjustwidth}
\textbf{(ii)}
\begin{adjustwidth}{2em}{0em}\vspace{-\baselineskip}
Für $P=1500 \cdot 8=12000$ Bit benötigt das Paket (inklusive Header von 160 Bit)
\[
\frac{12160 \text{Bit}}{10^6 \text{Bit/s}} + \frac{12160 \text{Bit}}{1000 \cdot 10^6 \text{Bit/s}} + \frac{12160 \text{Bit}}{10 \cdot 10^6 \text{Bit/s}} + 32,5 \cdot 10^{-6}\text{s} + 
2 \cdot 10^{-6}\text{s} = 13,42266 \cdot 10^{-3}\text{s}
\]
\end{adjustwidth}
\textbf{(iii)}
\begin{adjustwidth}{2em}{0em}\vspace{-\baselineskip}
Für $P=30000 \cdot 8=240000$ Bit benötigt das Paket (inklusive Header von 160 Bit)
\[
\frac{240160 \text{Bit}}{10^6 \text{Bit/s}} + \frac{240160 \text{Bit}}{1000 \cdot 10^6 \text{Bit/s}} + \frac{240160 \text{Bit}}{10 \cdot 10^6 \text{Bit/s}} + 32,5 \cdot 10^{-6}\text{s}
+ 2 \cdot 10^{-6}\text{s} = 264,45066 \cdot 10^{-3}\text{s}
\] \ \\
\end{adjustwidth}
\end{subexercise}
\begin{subexercise}
\textbf{(i)}
\begin{adjustwidth}{2em}{0em}\vspace{-\baselineskip}
Die Nachricht wird in $\frac{30000}{75}=400$ Paketen verschickt und die Versendung benötigt demnach $400 \cdot 0,87126 \cdot 10^{-3}\text{s}=348,504$ms.
\end{adjustwidth}
\textbf{(ii)}
\begin{adjustwidth}{2em}{0em}\vspace{-\baselineskip}
Die Nachricht wird in $\frac{30000}{1500}=20$ Paketen verschickt und die Versendung benötigt demnach $20 \cdot 13,42266 \cdot 10^{-3}\text{s}=268,4532$ms.
\end{adjustwidth}
\textbf{(iii)}
\begin{adjustwidth}{2em}{0em}\vspace{-\baselineskip}
Die Nachricht wird in einem Paket verschickt und die Versendung benötigt demnach $264,45066$ms.
\end{adjustwidth}
\end{subexercise}
\end{exercise}


\begin{exercise}{4}
\begin{subexercise}

\end{subexercise}
\begin{subexercise}

\end{subexercise}
\begin{subexercise}

\end{subexercise}
\end{exercise}


\begin{exercise}{2}
\begin{subexercise}
\begin{center}
\begin{tabular}{c|cc|cc|cc}
\multirow{2}{*}{Protokoll} & \multicolumn{2}{c}{lokal} & \multicolumn{2}{c}{global} & \multicolumn{2}{c}{Ziel} \\
 & IP-Adresse & Port & IP-Adresse & Port & IP-Adresse & Port \\
\hline
TCP & 10.0.0.1 & 8051 & 137.226.12.228 & 8051 & 137.226.13.142 & 443 \\
UDP & 10.0.0.3 & 4711 & 137.226.12.228 & 4711 & 8.8.8.8 & 53 \\
UDP & 10.0.0.4 & 4711 & 137.226.12.228 & 4712 & 8.8.8.8 & 53 \\
\end{tabular}
\end{center}
\end{subexercise}
\begin{subexercise}
Die Tabelle müsste um einen Eintrag ergänzt werden, welcher eingehende Anfragen auf Port 80 an Port 8888 des Rechners B weiterleitet, also ein Eintrag der Form\\
\begin{center}
\begin{tabular}{c|cc|cc|cc}
\multirow{2}{*}{Protokoll} & \multicolumn{2}{c}{lokal} & \multicolumn{2}{c}{global} & \multicolumn{2}{c}{Ziel} \\
 & IP-Adresse & Port & IP-Adresse & Port & IP-Adresse & Port \\
\hline
TCP & 10.0.0.2 & 8888 & 137.226.12.228 & 80 & - & - \\
\end{tabular}
\end{center}
\end{subexercise}
\end{exercise}

\end{document}
