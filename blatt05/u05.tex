\documentclass{../exercisesheet}

\title{Datenkommunikation und Informationssysteme, Übung 5}
\author{
    Domenic Quirl \\ 354437
    \and
    Julian Schakib \\ 353889
    \and 
    Daniel Schleiz \\ 356092
}

\renewcommand{\Exercise}{Aufgabe}
\date{Übungsgruppe 14}

\usepackage{float}
%\usepackage{siunitx}
\usepackage{color}
\usepackage{multirow}

\begin{document}
\maketitle
\pointtable


\begin{exercise}{4}
\begin{subexercise}
Zu Beginn ist für $i \in \{1,...,4\}$ die Weiterleitungstabelle von Switch i leer:\\
\begin{center}
\begin{tabular}{c|c}
\multicolumn{2}{c}{Switch i}\\
\hline
Rechner & Port\\
\hline
  &  \\
\end{tabular}
\end{center}
\begin{enumerate}
\item A sendet an B
\begin{itemize}
\item Switch 1 empfängt die Nachricht auf Port 1.1 und legt den Eintrag $A,\ 1.1$ an. Da er noch keinen Eintrag für B hat, broadcastet er die Nachricht an 1.2, 1.3 und 1.4. 1.2 ist Rechner B, der die Nachricht sieht. 
\item 1.3 ist 2.1 an Hub 1, welcher die Nachricht an 2.2 und 2.3 broadcastet. An 2.2 sieht Rechner C die Nachricht, 2.3 ist 3.1 an Switch 2. Dieser legt also auch einen Eintrag $A,\ 3.1$ an, kennt B aber ebenfalls nicht und broadcastet die Nachricht daher an 3.2 und 3.3. 3.2 ist D, der dann die Nachricht sieht, 3.3 ist 6.1 an Switch 4. Auch hier wird ein Eintrag $A,\ 6.1$  für A angelegt, B ist nicht bekannt und die Nachricht wird daher gebroadcastet. Sie erreicht G über 6.2
\item 1.4 ist 4.1 an Switch 3. Dieser legt den Eintrag $A,\ 4.1$ an und broadcastet an 4.2 und 4.3, da auch er B nicht kennt. An 4.3 sieht sie E, 4.2 ist 5.1 an Hub 2, welcher die Nachricht noch an 5.2 broadcastet, wo F sie sieht.
\end{itemize}
Es haben also die Rechner $(A), B, C, D, E, F, G$ den Rahmen gesehen, und die Weiterleitungstabellen sind wie folgt:\\
\begin{center}
\begin{tabular}{c|c}
\multicolumn{2}{c}{Switch 1}\\
\hline
Rechner & Port\\
\hline
A & 1.1 \\
\end{tabular}
\begin{tabular}{c|c}
\multicolumn{2}{c}{Switch 2}\\
\hline
Rechner & Port\\
\hline
A & 3.1 \\
\end{tabular}
\begin{tabular}{c|c}
\multicolumn{2}{c}{Switch 3}\\
\hline
Rechner & Port\\
\hline
A & 4.1 \\
\end{tabular}
\begin{tabular}{c|c}
\multicolumn{2}{c}{Switch 4}\\
\hline
Rechner & Port\\
\hline
A & 6.1 \\
\end{tabular}
\end{center}
\item C sendet an A
\begin{itemize}
\item Hub 1 empfängt die Nachricht auf Port 2.2 und broadcastet sie an 2.1 und 2.3.
\item 2.3 ist 3.1 an Switch 2. Dieser hat einen Eintrag für A für denselben Port, verwirft also den Rahmen. Trotzdem legt er den Eintrag $C,\ 3.1$ an.
\item 2.1 ist 1.3 an Switch 1. Dieser hat einen Eintrag für A, nämlich 1.1. Er leitet die Nachricht also an Port 1.1 weiter, wo A sie sieht, und legt den Eintrag $C,\ 1.3$ an.
\end{itemize}
Es haben also nur die Rechner $(C), A$ den Rahmen gesehen. Die Weiterleitungstabellen der Switches sind wie folgt:\\
\begin{center}
\begin{tabular}{c|c}
\multicolumn{2}{c}{Switch 1}\\
\hline
Rechner & Port\\
\hline
A & 1.1 \\
C & 1.3 \\
\end{tabular}
\begin{tabular}{c|c}
\multicolumn{2}{c}{Switch 2}\\
\hline
Rechner & Port\\
\hline
A & 3.1 \\
C & 3.1 \\
\end{tabular}
\begin{tabular}{c|c}
\multicolumn{2}{c}{Switch 3}\\
\hline
Rechner & Port\\
\hline
A & 4.1 \\
  &   \\
\end{tabular}
\begin{tabular}{c|c}
\multicolumn{2}{c}{Switch 4}\\
\hline
Rechner & Port\\
\hline
A & 6.1 \\
  &   \\
\end{tabular}
\end{center}
\item G sendet an A
\begin{itemize}
\item Switch 4 empfängt die Nachricht auf Port 6.2 und legt den Eintrag $G,\ 6.2$ an. Da er bereits den Eintrag 6.1 für A besitzt, leitet er sie an diesen Port weiter.
\item Dort empfängt sie Switch 2 auf Port 3.3, welcher ebenfalls zunächst einen Eintrag $G,\ 3.3$ anlegt. Da er einen Eintrag für A hat, sendet er die Nachricht weiter an Port 3.1.
\item Dies ist Port 2.3 an Hub 1. Dieser broadcastet die Nachricht an 2.1 und 2.2. An 2.2 wird sie von C gesehen.
\item 2.1 ist 1.3 an Switch 1. Auch hier wird ein Eintrag $G,\ 1.3$ angelegt. Switch 1 hat bereits einen Eintrag für A für 1.1, leitet den Rahmen also nur an diesen Port weiter, wo A ihn empfängt.
\end{itemize}
Es haben also die Rechner $(G),C,A$ den Rahmen gesehen. Die Weiterleitungstabellen sind wie folgt:\\
\begin{center}
\begin{tabular}{c|c}
\multicolumn{2}{c}{Switch 1}\\
\hline
Rechner & Port\\
\hline
A & 1.1 \\
C & 1.3 \\
G & 1.3 \\
\end{tabular}
\begin{tabular}{c|c}
\multicolumn{2}{c}{Switch 2}\\
\hline
Rechner & Port\\
\hline
A & 3.1 \\
C & 3.1 \\
G & 3.3 \\
\end{tabular}
\begin{tabular}{c|c}
\multicolumn{2}{c}{Switch 3}\\
\hline
Rechner & Port\\
\hline
A & 4.1 \\
  &  \\
  &   \\
\end{tabular}
\begin{tabular}{c|c}
\multicolumn{2}{c}{Switch 4}\\
\hline
Rechner & Port\\
\hline
A & 6.1 \\
G & 6.2 \\
  &   \\
\end{tabular}
\end{center}
\end{enumerate}
\end{subexercise}
\begin{subexercise}
Seien Hub 3 und Hub 4 die Hubs, die Switch 1 und 3 ersetzen. Wenn C an F sendet, broadcastet Hub 1 an Hub 2 und 3 und dann diese an alle Rechner sowie beide an Hub 4. Es kommt also zur Kollision, da der Rahmen von Hub 1 aus von beiden Seiten Richtung Hub 4 gesendet wird.
\end{subexercise}
\begin{subexercise}
Solange die Switches haben keinen Eintrag für F haben, ändert sich nichts, da sie dann genau wie ein Hub die eingehende Nachricht broadcasten. Bei Switch 1 macht auch ein Eintrag für F nur den Unterschied, dass A und B den Rahmen nicht mehr sehen. Hat Switch 3 allerdings einen Eintrag für F auf 4.2, würde er eine eingehende Nachricht an F auf diesem Port verwerfen. So hängt es dann vom zeitlichen Ablauf im Netzwerk ab, ob es noch zu einer Kollision kommt, im Gegensatz zu vorher tritt diese dann nicht mehr auf jeden Fall irgendwo auf, sondern nur, wenn sie Switch 3 auf 4.2 noch nicht erreicht hat, bevor dieser sie von 4.1 zu 4.2 weiterleitet. Eine Kollision kann also nur noch in Hub 2 auftreten.
\end{subexercise}
\end{exercise}


\begin{exercise}{5}
\begin{subexercise}
Berechne zunächst die Latenzen (Länge geteilt durch die Ausbreitungsgeschwindigkeit) und die maximalen Datenraten zwischen den Zwischenknoten:\\
\begin{table}[h]
\centering
\begin{tabular}{c|c|c}
 & Latenz & max. Datenrate \\ \hline
$S\rightarrow R_1$      & 2,5$\mu$s & 1 Mbit/s \\ \hline
$R_1 \rightarrow R_2$ & 25$\mu$s & 1000 Mbit/s \\ \hline
$R_2 \rightarrow D$     & 5$\mu$s & 10 Mbit/s
\end{tabular}
\end{table}\\
(Bei NRZ wird pro Schritt ein Bit kodiert, also in dem Fall  entspricht 1 MBaud gerade 1 Mbit/s. Bei 4B/5B werden 4 Bits in 5 Schritten übertragen, d.h. $1250 \cdot 0,8$ Mbit/s. Für den
Manchester Leitungscode werden zwei Schritte benötigt, um ein Bit zu übertragen, also $20 \cdot 0,5$ Mbit/s.)\\ \ \\
\textbf{(i)}
\begin{adjustwidth}{2em}{0em}\vspace{-\baselineskip}
Für $P=75 \cdot 8=600$ Bit benötigt das Paket (inklusive Header von 160 Bit)
\[
\frac{760 \text{Bit}}{10^6 \text{Bit/s}} + \frac{760 \text{Bit}}{1000 \cdot 10^6 \text{Bit/s}} + \frac{760 \text{Bit}}{10 \cdot 10^6 \text{Bit/s}} + 32,5 \cdot 10^{-6}\text{s} + 
2 \cdot 10^{-6}\text{s} = 0,87126 \cdot 10^{-3}\text{s}
\]
(Benötigte Zeit zur Übertragung der jeweiligen Leitungen plus die summierten Latenzen plus die Verarbeitungszeiten der Zwischenstationen $R_i$.)
\end{adjustwidth}
\textbf{(ii)}
\begin{adjustwidth}{2em}{0em}\vspace{-\baselineskip}
Für $P=1500 \cdot 8=12000$ Bit benötigt das Paket (inklusive Header von 160 Bit)
\[
\frac{12160 \text{Bit}}{10^6 \text{Bit/s}} + \frac{12160 \text{Bit}}{1000 \cdot 10^6 \text{Bit/s}} + \frac{12160 \text{Bit}}{10 \cdot 10^6 \text{Bit/s}} + 32,5 \cdot 10^{-6}\text{s} + 
2 \cdot 10^{-6}\text{s} = 13,42266 \cdot 10^{-3}\text{s}
\]
\end{adjustwidth}
\textbf{(iii)}
\begin{adjustwidth}{2em}{0em}\vspace{-\baselineskip}
Für $P=30000 \cdot 8=240000$ Bit benötigt das Paket (inklusive Header von 160 Bit)
\[
\frac{240160 \text{Bit}}{10^6 \text{Bit/s}} + \frac{240160 \text{Bit}}{1000 \cdot 10^6 \text{Bit/s}} + \frac{240160 \text{Bit}}{10 \cdot 10^6 \text{Bit/s}} + 32,5 \cdot 10^{-6}\text{s}
+ 2 \cdot 10^{-6}\text{s} = 264,45066 \cdot 10^{-3}\text{s}
\] \ \\
\end{adjustwidth}
\end{subexercise}
\begin{subexercise}
\textbf{(i)}
\begin{adjustwidth}{2em}{0em}\vspace{-\baselineskip}
Die Nachricht wird in $\frac{30000}{75}=400$ Paketen verschickt und die Versendung benötigt demnach $400 \cdot 0,87126 \cdot 10^{-3}\text{s}=348,504$ms.
\end{adjustwidth}
\textbf{(ii)}
\begin{adjustwidth}{2em}{0em}\vspace{-\baselineskip}
Die Nachricht wird in $\frac{30000}{1500}=20$ Paketen verschickt und die Versendung benötigt demnach $20 \cdot 13,42266 \cdot 10^{-3}\text{s}=268,4532$ms.
\end{adjustwidth}
\textbf{(iii)}
\begin{adjustwidth}{2em}{0em}\vspace{-\baselineskip}
Die Nachricht wird in einem Paket verschickt und die Versendung benötigt demnach $264,45066$ms.
\end{adjustwidth}
\end{subexercise}
\end{exercise}


\begin{exercise}{4}
Damit die Adressen nicht zu lang werden, wird im Folgenden, wenn die Betrachtung der Binärdarstellung eines gewissen Teils notwendig ist, nur der relevante Teil binär dargestellt. \\
\begin{subexercise}
Der IP-Adressbereich \texttt{137.226.40.0/21} impliziert eine Subnetzmaske mit 21 Einsen, d.h. die Subnetzmaske \texttt{255.255.248.0 = 255.255.11111000.0} (und \texttt{137.226.40.0 = 137.226.00101000.0}).
\begin{itemize}
\item Verundet man IP 1 mit der Subnetzmaske, so erhält man die Adresse \texttt{137.226.48.0 = \\ 137.226.00110000.0}. Die Adresse liegt also nicht im gegebenen Adressbereich, da 
	\texttt{00110 $\neq$ 00101}.
\item Man sieht direkt, dass IP 2 nicht im Adressbereich liegt, weil schon im ersten 8 Bit Teil der Adresse ein Unterschied vorliegt und dieser Teil offensichtlich bei der Subnetzmaske verundet
	wird. (\texttt{136 $\neq$ 137}).
\end{itemize}
\end{subexercise}
\begin{subexercise}
\begin{itemize}
\item Um 900 Rechner in LAN 1 zu adressieren, benötigt man 10 Bit ($2^9 - 2 = 510 < 900 < 1022 = 2^{10}-2$). Somit kriegt LAN 1 den Adressbereich \texttt{137.226.40.0/22} mit
	Subnetzmaske \texttt{255.255.252.0}. (So klein wie möglich, da 11 Bit zur Verfügung standen.) Das Subnetz erhält als Netz-ID die niedrigste Adresse des Subnetzes, also 
	\texttt{137.226.40.0}. (Hosts: \texttt{137.226.001010xx.xxxxxxxx})
	\\ Der restliche Adressbereich umfasst \texttt{137.226.44.0/22}.
\item Um 200 Rechner in LAN 2 zu adressieren, benötigt man 8 Bit ($2^7 - 2 = 126 < 200 < 254 = 2^8-2$). Der kleinstmögliche Adressbereich für LAN 2 wäre dann 
	\texttt{137.226.44.0/24} mit Subnetzmaske \texttt{255.255.255.0} und Netz-ID \texttt{137.226.44.0}. (Hosts: \texttt{137.226.00101100.xxxxxxxx})
\item Um 500 Rechner in LAN 3 zu adressieren, benötigt man 9 Bit ($2^8 - 2 = 254 < 500 < 510 = 2^9-2$). Der kleinstmögliche Adressbereich für LAN 3 wäre dann 
	\texttt{137.226.46.0/23} mit Subnetzmaske \texttt{255.255.254.0} und Netz-ID \texttt{137.226.46.0}. (Hosts: \texttt{137.226.0010101x.xxxxxxxx})
\item Um 75 Rechner adressieren zu können benötigt man 7 Bit. Der kleinst mögliche Adressbereich für LAN 4 wäre dann \texttt{137.226.45.0/25} mit Subnetzmaske 
	\texttt{255.255.255.128} und Netz-ID \texttt{137.226.45.0}. (Hosts: \texttt{137.226.00101001.0xxxxxxx}).
\end{itemize}
\ \\Nach der Einteilung ist noch der Adressbereich \texttt{137.226.45.128/25} frei.
\end{subexercise}
\begin{subexercise}
Die höchste Adresse eines Subnetzes ist für Broadcast reserviert, weshalb diese nicht vergeben wird. Nach den Vergaberegeln der Aufgabenstellung ergebt sich folgende Verteilung von
IP-Adressen:
\begin{itemize}
\item In LAN 1 erhält A.if1 \texttt{137.226.40.1}, h1 kriegt \texttt{137.226.43.254} und h2 kriegt \texttt{137.226.43.253}
\item A.if2: \texttt{137.226.44.1}, B.if1: \texttt{137.226.44.2}, h3: \texttt{137.226.44.254}
\item B.if2: \texttt{137.226.46.1}, h4: \texttt{137.226.47.254}
\item B.if3: \texttt{137.226.45.1}, h5: \texttt{137.226.45.126}
\end{itemize}
\end{subexercise}
\end{exercise}


\begin{exercise}{2}
\begin{subexercise}
\begin{center}
\begin{tabular}{c|cc|cc|cc}
\multirow{2}{*}{Protokoll} & \multicolumn{2}{c}{lokal} & \multicolumn{2}{c}{global} & \multicolumn{2}{c}{Ziel} \\
 & IP-Adresse & Port & IP-Adresse & Port & IP-Adresse & Port \\
\hline
TCP & 10.0.0.1 & 8051 & 137.226.12.228 & 8051 & 137.226.13.142 & 443 \\
UDP & 10.0.0.3 & 4711 & 137.226.12.228 & 4711 & 8.8.8.8 & 53 \\
UDP & 10.0.0.4 & 4711 & 137.226.12.228 & 4712 & 8.8.8.8 & 53 \\
\end{tabular}
\end{center}
\end{subexercise}
\begin{subexercise}
Die Tabelle müsste um einen Eintrag ergänzt werden, welcher eingehende Anfragen auf Port 80 an Port 8888 des Rechners B weiterleitet, also ein Eintrag der Form\\
\begin{center}
\begin{tabular}{c|cc|cc|cc}
\multirow{2}{*}{Protokoll} & \multicolumn{2}{c}{lokal} & \multicolumn{2}{c}{global} & \multicolumn{2}{c}{Ziel} \\
 & IP-Adresse & Port & IP-Adresse & Port & IP-Adresse & Port \\
\hline
TCP & 10.0.0.2 & 8888 & 137.226.12.228 & 80 & - & - \\
\end{tabular}
\end{center}
\end{subexercise}
\end{exercise}

\end{document}
