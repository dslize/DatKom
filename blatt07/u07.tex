\documentclass{../exercisesheet}

\title{Datenkommunikation und Informationssysteme, Übung 7}
\author{
    Domenic Quirl \\ 354437
    \and
    Julian Schakib \\ 353889
    \and 
    Daniel Schleiz \\ 356092
}

\renewcommand{\Exercise}{Aufgabe}
\date{Übungsgruppe 14}

\usepackage{float}
%\usepackage{siunitx}
\usepackage{color}
\usepackage{multirow}
\usepackage{float}

\begin{document}
\maketitle
\pointtable


\begin{exercise}{1.5}
\begin{subexercise}

\end{subexercise}
\begin{subexercise}

\end{subexercise}
\begin{subexercise}

\end{subexercise}
\end{exercise}


\begin{exercise}{4}
\begin{subexercise}

\end{subexercise}
\begin{subexercise}

\end{subexercise}
\begin{subexercise}

\end{subexercise}
\begin{subexercise}

\end{subexercise}
\end{exercise}


\begin{exercise}{4}
\begin{center}
\begin{tabular}{|c||c|c|c|c|c|c|c|c|c|c|}
\hline
 & \multicolumn{10}{c|}{Schritt} \\
\hline
Router & 0 & 1 & 2 & 3 & 4 & 5 & 6 & 7 & 8 & 9 \\
\hline
\hline
A & \fbox{0,-} & 0,- & 0,- & 0,- & 0,- & 0,- & 0,- & 0,- & 0,- & 0,- \\
\hline
B & $\infty$ & 8,A & \fbox{8,A} & 8,A & 8,A & 8,A & 8,A & 8,A & 8,A & 8,A \\
\hline
C & $\infty$ & $\infty$ & 9,H & \fbox{9,H} & 9,H & 9,H & 9,H & 9,H & 9,H & 9,H \\
\hline
D & $\infty$ & $\infty$ & $\infty$ & $\infty$ & \fbox{11,C} & 11,C & 11,C & 11,C & 11,C & 11,C \\
\hline
E & $\infty$ & $\infty$ & $\infty$ & $\infty$ & 12,C & \fbox{12,C} & 12,C & 12,C & 12,C & 12,C \\
\hline
F & $\infty$ & $\infty$ & 13,H & 13,H & 13,H & 13,H & \fbox{13,H} & 13,H & 13,H & 13,H \\
\hline
G & $\infty$ & $\infty$ & $\infty$ & $\infty$ & $\infty$ & $\infty$ & 15,E & 15,E & \fbox{15,E} & 15,E \\
\hline
H & $\infty$ & \fbox{6, A} & 6, A & 6, A & 6, A & 6, A & 6, A & 6, A & 6, A & 6, A \\
\hline
I & $\infty$ & $\infty$ & $\infty$ & $\infty$ & $\infty$ & $\infty$ & 14,E & \fbox{14,E} & 14,E & 14,E \\
\hline
\end{tabular}
\end{center}
\begin{description}
\item[A -> I:] Der durch den Dijkstra-Algorithmus ermittelte kürzeste Pfad von A nach I ist (A,H),(H,C),(C,E),(E,I) mit Kosten 14.

Es existieren außerdem folgende Pfade von A nach I mit gleichen Kosten:
\begin{itemize}
\item (A,B),(B,C),(C,E),(E,I)
\end{itemize}
\end{description}
\end{exercise}

\begin{exercise}{5.5}

\end{exercise}

\end{document}

