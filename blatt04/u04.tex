\documentclass{../exercisesheet}

\title{Datenkommunikation und Informationssysteme, Übung 4}
\author{
    Domenic Quirl \\ 354437
    \and
    Julian Schakib \\ 353889
    \and 
    Daniel Schleiz \\ 356092
}

\renewcommand{\Exercise}{Aufgabe}
\date{Übungsgruppe 14}

\usepackage{float}
%\usepackage{siunitx}
\usepackage{color}

\begin{document}
\maketitle
\pointtable


\begin{exercise}{1}
	Es ergeben sich folgende Hamming Codes (eingefügte Paritätsbits rot markiert):
	\begin{itemize}
	\item \textcolor{red}{01}1\textcolor{red}{1}000\textcolor{red}{1}0101001\textcolor{red}{1}00010
	\item \textcolor{red}{11}0\textcolor{red}{1}111\textcolor{red}{0}0110101\textcolor{red}{1}10011
	\end{itemize}
\end{exercise}

\begin{exercise}{4}
\begin{subexercise}
	Zur Berechnung der maximalen Nutzdatenrate soll die Übertragung eines Pakets mit der maximalen Menge an Nutzdaten pro Rahmen, also 343 Byte, betrachtet werde. Gegeben ist
	außerdem die maximale Datenrate von 1000 Bit/s und eine Latenz von 0,003s. \\
	Mit dem Header von 16 Byte werden also $343+16=359$ Byte $= 2872$ Bit von $A$ nach $B$ übertragen. Dazu werden $\frac{2872 \text{Bit}}{1000 \text{Bit/s}} +$ 0,003s 
	= 2,875s benötigt. Anschließend wird, da keine Bitfehler passieren, ein ACK Rahmen von 16 Byte = 128 Bit in $\frac{128 \text{Bit}}{1000 \text{Bit/s}} +$ 0,003s = 0,131s von
	$B$ nach $A$ übertragen. Insgesamt benötigte man also 2,875 + 0,131 = 3,006s für die Übertragung des Pakets und die Bestätigung, es ergibt sich also die maximale
	Nutzdatenrate von $\frac{343\text{Bit}}{\text{3,006s}}\approx$ 114 Bit/s = 0,114 Mbit/s.
\end{subexercise}
\begin{subexercise}
	Als direkte Maßnahme könnte man die maximale Größe von Nutzdaten pro Rahmen vergrößern. Dies würde zu einer Steigerung der (maximalen) Nutzdatenrate führen, da 
	insgesamt weniger Pakete verschickt werden müssten bei großen Datenmengen und somit weniger Quittierungen durch das Zurücksenden eines ACK Rahmens stattfinden
	würden. Für ein Byte Nutzdaten muss dann also im Schnitt weniger auf ein ACK "gewartet" werden.
\end{subexercise}
\begin{subexercise}
	
\end{subexercise}
\end{exercise}

\begin{exercise}{2.5}
\begin{subexercise}
	
\end{subexercise}
\begin{subexercise}
	
\end{subexercise}
\end{exercise}

\begin{exercise}{1.5}
\begin{subexercise}
	
\end{subexercise}
\begin{subexercise}
	
\end{subexercise}
\begin{subexercise}
	
\end{subexercise}
\end{exercise}

\begin{exercise}{6}
\begin{subexercise}
	
\end{subexercise}
\begin{subexercise}
	
\end{subexercise}
\begin{subexercise}
	
\end{subexercise}
\end{exercise}


\end{document}
