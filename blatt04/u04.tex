\documentclass{../exercisesheet}

\title{Datenkommunikation und Informationssysteme, Übung 4}
\author{
    Domenic Quirl \\ 354437
    \and
    Julian Schakib \\ 353889
    \and 
    Daniel Schleiz \\ 356092
}

\renewcommand{\Exercise}{Aufgabe}
\date{Übungsgruppe 14}

\usepackage{float}
%\usepackage{siunitx}
\usepackage{color}

\begin{document}
\maketitle
\pointtable


\begin{exercise}{1}
	Es ergeben sich folgende Hamming Codes (eingefügte Paritätsbits rot markiert):
	\begin{itemize}
	\item \textcolor{red}{01}1\textcolor{red}{1}000\textcolor{red}{1}0101001\textcolor{red}{1}00010
	\item \textcolor{red}{11}0\textcolor{red}{1}111\textcolor{red}{0}0110101\textcolor{red}{1}10011
	\end{itemize}
\end{exercise}

\begin{exercise}{4}
\begin{subexercise}
	Zur Berechnung der maximalen Nutzdatenrate soll die Übertragung eines Pakets mit der maximalen Menge an Nutzdaten pro Rahmen, also 343 Byte, betrachtet werde. Gegeben ist
	außerdem die maximale Datenrate von 1000000 Bit/s und eine Latenz von 0,003s. \\
	Mit dem Header von 16 Byte werden also $343+16=359$ Byte $= 2872$ Bit von $A$ nach $B$ übertragen. Dazu werden $\frac{2872 \text{Bit}}{1000000 \text{Bit/s}} +$ 0,003s 
	= 0,005872s benötigt. Anschließend wird, da keine Bitfehler passieren, ein ACK Rahmen von 16 Byte = 128 Bit in $\frac{128 \text{Bit}}{1000000 \text{Bit/s}} +$ 0,003s = 0,003128s von
	$B$ nach $A$ übertragen. Insgesamt benötigte man also 0,005872 + 0,003128 = 0,009s für die Übertragung des Pakets und der Bestätigung, es ergibt sich also die maximale
	Nutzdatenrate von $\frac{343\cdot 8\text{Bit}}{\text{0,009s}}\approx$ 304889 Bit/s = 304,889 kBit/s $\approx$ 305 kBit/s.
\end{subexercise}
\begin{subexercise}
	Als direkte Maßnahme könnte man die maximale Größe von Nutzdaten pro Rahmen (also die Rahmengröße) vergrößern. Dies würde zu einer Steigerung der (maximalen) Nutzdatenrate führen, da 
	insgesamt weniger Pakete verschickt werden müssten bei großen Datenmengen und somit weniger Quittierungen durch das Zurücksenden eines ACK Rahmens stattfinden
	würden. Für ein Byte Nutzdaten muss dann also im Schnitt weniger auf ein ACK "gewartet" werden.
\end{subexercise}
\begin{subexercise}
	Gehe bei einer erfolgreichen Übertragung eines Pakets von der Nutzdatenrate von (a) aus, d.h. 305 kBit/s. Eine Übertragung schlägt fehl, wenn entweder der Rahmen mit
	Nutzdaten und Header (359 Byte) einen Bitfehler hat, oder falls das darauf folgende ACK (16 Byte) einen Bitfehler hat und somit als NACK erkannt wird. Die Wahrscheinlichkeit
	für ein Bit, fehlerfrei zu sein, ist $1-10^{-4}=0.9999$. Die Wahrscheinlichkeit, dass alle der 359 + 16 Byte = 375 Byte = 3000 Bit fehlerfrei sind, liegt demnach bei
	$0.9999^{3000}\approx 0.74$. Dementsprechend liegt die Wahrscheinlichkeit eines Fehlers bei der Übertragung bei $1-0.9999^{3000}\approx 0.26$. Im Falle eines Fehlers
	müssen erneut der Rahmen mit Nutzdaten und Header und das anschließende ACK gesendet werden, also die doppelte Datenmenge. Somit halbiert sich im Fehlerfall
	die Nutzdatenrate auf 152.5 Mbit/s. Nun kann bei der ernteuten Übertragung wieder ein Fehler auftreten, es würde die dreifache Datenmenge übertragen, die Datenrate drittelt sich.
	Die Wahrscheinlichkeit für diesen Fall (im dritten Anlauf soll die Übertragung klappen) beträgt $0.26 \cdot 0.26 \cdot 0.74$. Nach diesem Prinzip lässt sich für die mittlere
	Nutzdatenrate $\overline{d}$ die folgende Formel konstruieren (j gibt also die benötigte Anzahl an Übertragungswiederholungen an):
	\[
	\overline{d} =\sum_{j=0}^{\infty}0.74 \cdot 0.26^j \cdot \frac{305}{j+1} = 261.532 \text{ [kBit/s]}
	\]
	%Die mittlere Nutzdatenrate liegt somit bei $0.9999^{3000} \cdot 0,114 + (1-0.9999^{3000}) \cdot 0.057 \approx 0.099$ Mbit/s.
\end{subexercise}
\begin{subexercise}
	Da die Aufgabenstellung suggeriert, dass beliebig viele empfangene Rahmen auf einmal verwaltet werden können, bietet es sich an, \textit{Selective Repeat} zu verwenden,
	da im Vergleich zu Go-Back-N die auf einen fehlerhaften Rahmen folgenden Rahmen nicht verworfen werden und erneut übertragen werden müssen.
\end{subexercise}
\end{exercise}

\begin{exercise}{2.5}
\begin{subexercise}
Für eine Auslastung von 100\% muss die Nutzdatenrate der maximal möglichen Datenrate entsprechen, also 30 Mbit/s betragen. Da $30\ \text{Mbit/s} = 30000000\ \text{Bit/s} = 3750000\ \text{Byte/s}$ entspricht dies dem Senden eines Rahmens alle $\frac{8000\ \text{Byte}}{3750000\ \text{Byte/s}} = 2.1\bar{3} * 10^{-3}$s bzw. dem Senden von 468.75 Rahmen/s. Die Latenz von 250ms entspricht demnach 117.1875 gesendeten Rahmen, die doppelte Latenz also 234.375 Rahmen.

Das bedeutet, dass ein ACK für den ersten gesendeten Rahmen, wenn der Sender kontinuierlich mit der gewünschten Datenrate weiter sendet, frühestens wieder den Sender erreicht, wenn dieser bereits ca. 234 weitere Rahmen gesendet hat. Dies ist natürlich nur möglich, wenn die Fenstergröße mindestens 235 beträgt, da der Sender sonst im Senden innehalten und zunächst auf das ACK für seinen ersten Rahmen warten müsste, um wieder weitere Rahmen senden zu können. Dann würde aber eine geringere als die gewünschte Auslastung erreicht, sodass insgesamt das Fenster mindestens diese Größe von 235 haben muss.
\end{subexercise}
\begin{subexercise}
	Eine Auslastung von 10\% entspricht einer Nutzdatenrate von 3 Mbit/s. Unter Vernachlässigung der Sendezeit des Senders ergäbe sich folgendes Bild:
	
	Die Gesamtzeit, bis der Sender ein ACK für einen gesendeten Rahmen erhalten kann (Latenz + Verarbeitungszeit + Latenz), beträgt 750ms. Wenn wir annehmen, dass in dieser Zeit genau so viele Rahmen verschickt werden sollen, wie die Fenstergröße erlaubt, um die benötigte Rahmengröße zu minimieren, ergibt das eine Übertragungsrate von 200 Rahmen/s. Um mit diesen Rahmen die gewünschten 3 Mbit zu übertragen, muss jeder Rahmen $\frac{3\ \text{Mbit/s}}{200\ \text{Rahmen/s}} = 15$ KBit/Rahmen = 1875 Byte/Rahmen enthalten.
	\\
	
	Mit Sendezeit des Senders ist diese, wenn $R$ unsere Mindestgröße der Rahmen beschreibt, $t_s = \frac{R}{3\ \text{Mbit/s}}$. Daraus ergibt sich die maximale Rate übertragbarer Rahmen pro Sekunde unter Berücksichtigung der Fenstergröße zu $S = \frac{150}{t_s + 0.75 \text{s}}$ Rahmen/s. Um so die gewünschte Auslastung zu erzielen, muss die Rahmengröße mindestens
	\begin{equation*}
	\begin{split}
	&R = \frac{3\ \text{Mbit/s}}{S} = \frac{3\ \text{Mbit/s} * t_s + 0.75 \text{s}}{150} = \frac{R + 2.25 \text{Mbit}}{150} = \frac{R}{150} + 15\ \text{Kbit} \\
	\leftrightarrow &R \approx 15.1007\ \text{Kbit}
	\end{split}
	\end{equation*}
betragen (dies ergibt $S = 198.\bar{6}$ Rahmen/s).
\end{subexercise}
\end{exercise}

\begin{exercise}{1.5}
\begin{subexercise}
	Bei 4 Bit für Sequenz- und Quittungsnummer ist die maximale Größe des Sendefensters $2^4 = 16$. Der Sender darf also noch 12 weitere Rahmen verschicken bis er auf eine Quittung warten muss, je nach MODULUS zum Beispiel 7 - 16 und 0 - 1.
\end{subexercise}
\begin{subexercise}
	Mit $N(R) = 6$ bestätigt der Empfänger den Erhalt aller Rahmen bis $N(S) = 5$. Es können also drei zusätzliche Rahmen (insgesamt 15) verschickt werden, da sich das Fenster über 3, 4, 5 nach vorne verschiebt, je nach MODULUS zum Beispiel 7 - 16 und 0 - 4.
\end{subexercise}
\begin{subexercise}
	Das Reject mit $N(R) = 5$ bestätigt alle Rahmen bis $N(S) = 4$ und fordert eine Neuübertragung der Rahmen ab $N(S) = 5$ an. Es können also zwei weitere Rahmen übertragen werden - damit insgesamt 14 -, zwei davon sollten allerdings die Rahmen mit $N(S) = 5$ und $N(S) = 6$ sein.
\end{subexercise}
\end{exercise}

\begin{exercise}{6}
\begin{subexercise}
\textbf{(i)}
\begin{adjustwidth}{2em}{0em}\vspace{-\baselineskip}
	%Zwischen $S_1$ und $S_2$ liegt eine Latenz von $\frac{90}{3 \cdot 10^8}= 3\cdot 10^{-7}$s vor, zwischen $S_2$ bzw. $S_1$ und $D$ dementsprechend $1.5 \cdot 10^{-7}$s,
	%da nur die halbe Strecke vorliegt.
	Bei einer Datenrate von 10MBit/s sendet $S_1$ die 1400 Byte = 0.0014 MByte = 0.0112 MBit in $\frac{0.0112}{10}=0.00112$s, also von $t_1=1$s bis $t_2=1.00112$s. Da $S_2$ ebenfalls nur an
	$D$ senden möchte, kommt es nur zur Kollision, falls sich Rahmen im Punkt $D$ überlagern. Weil $S_1$ und $S_2$ genau gleich entfernt von $D$ sind und die gleiche Menge an Daten
 	senden möchten, kommt es zur Kollision, falls $S_2$ im gleichen Sendezeitraum wie $S_1$ oder früher senden möchte, d.h. von $t=0.99888$s bis $t_2$. (Die Latenzen heben sich auf, da
	gleiche Entfernung.)\\
\end{adjustwidth}
\textbf{(ii)}
\begin{adjustwidth}{2em}{0em}\vspace{-\baselineskip}
	$S_1$ möchte bei $t=1000$ms senden, jedoch beginnt erst bei $t_1=1000.5$ms ein Zeitslot, da $1.5 \cdot 667 = 1000.5$. Somit würde $S_1$ ab $t_1$ senden und alle
	darauf folgenden Slots belegen, bis die 1400 Byte übertragen sind. Aufgrund gleicher physikalischer Bedingungen wie in (a) sendet $S_1$ 1.12ms lang. 
	Da ein Slot 1.5ms lang ist und $1.12<1.5$, werden die Daten innerhalb dieses Slots übertragen. Es kommt also zur Kollision, wenn $S_1$ und $S_2$ um den selben Slot konkurrieren,
	d.h. wenn $S_2$ einen Sendewunsch nach 999ms (kriegt diesen Slot also gerade nicht mehr) und vor 1000.5ms hat (und somit gerade den Slot von $S_1$ noch kriegt).\\
	%Der letzte Slot,
	%den $S_1$ dabei belegt, liegt bei $t_2=2119.5$ms, da $S_1$ beim Zeitpunkt $1000.5+1120=2120.5$ms fertig mit Senden ist . (Die letzten 0.5ms des Slots bleiben ungenutzt.) Da
	%wieder $S_2$ Daten mit gleicher Größe senden möchte und beide die gleiche Latenz zu $D$ aufweisen, kommt es zur Kollision, falls $S_2$ einen Sendewunsch 
	%ab $t=0$s bis
		%nach 999ms (wird diesem Slot gerade nicht mehr zugeteilt, erster Zeitslot wäre dann bei $t_1$) 
	 %vor bzw. bei $t_2$ hat, also dem letzten von $S_1$ belegten Slot.\\ 
\end{adjustwidth}
\textbf{(iii)}
\begin{adjustwidth}{2em}{0em}\vspace{-\baselineskip}
	Es kommt CSMA/CD zum Einsatz. Zwischen $S_1$ und $S_2$ liegt eine Latenz von $\frac{90}{2 \cdot 10^8}= 4.5\cdot 10^{-7}$s vor. Bei CSMA kann eine Station nur senden, 
	falls die Station auf der Leitung gerade nichts hört. Da eine Latenz zwischen $S_1$ und $S_2$ herrscht, kommt es zur Kollision, falls $S_2$ anfängt zu senden, bevor
	das Signal durch die Latenz verzögert bei $S_2$ ankommt bzw. $S_2$ anfängt zu senden, sodass durch die Latenz das Signal bei $S_1$ bei 1s noch nicht angekommen ist.
	Es folgt also, dass es zur Kollision kommt, falls $S_2$ von $1-4.5\cdot 10^{-7}$s bis $1+4.5\cdot 10^{-7}$s senden möchte. Möchte $S_2$ davor oder danach senden, so
	wird im ersten Fall $S_1$ das Senden verboten und es kommt nicht zur Kollision bzw. $S_2$ wird es im zweiten Fall verboten zu Senden. (Sofern die Datenübertragung noch nicht abgeschlossen.)\\
\end{adjustwidth}
\textbf{(iv)}
\begin{adjustwidth}{2em}{0em}\vspace{-\baselineskip}
	Es kann zu keiner Kollision kommen, da es sich bei Token Ring um ein geregeltes Verfahren handelt, bei dem nur gesendet werden kann, falls ein Frei-Token ankommt, 
	welches dann das Token belegt. $S_2$ muss also warten, bis wieder ein Frei-Token ankommt bzw. hat das Frei-Token kurz vor $S_1$ belegt, woraufhin $S_1$ warten muss.
\end{adjustwidth}
\end{subexercise}
\begin{subexercise}
	Auf die 90m betrachtet, also die maximale Ausbreitung, darf eine Sendung nicht vorüber sein, solange das Round Trip Delay noch nicht abgelaufen ist, also das Doppelte
	der Latenz von $4.5\cdot 10^{-7}$s. Der Round Trip Delay beträgt also $9\cdot 10^{-7}$s. Während dieser Zeit können mit 10 MBit/s Datenrate Daten der
	Größe $9\cdot 10^{-7}s \cdot 10 \cdot 10^6$ Bit/s = 9 Bit übertragen werden. Somit muss die minimale Rahmenlänge 9 Bit betragen.\\
	Die minimale Rahmenlänge ist notwendig, damit zuverlässig Kollisionen detektiert werden können. Bei einer Kollision wird bei CSMA/CD ein Jamming Signal ausgesendet. Damit
	der Sender weiß, dass er mit dem Jamming Signal gemeint ist, darf die Sendung noch nicht beendet sein.
\end{subexercise}
\begin{subexercise}
	Einerseits ist steigt bei großer Ausdehnung die \textit{minimale Rahmengröße} an, da sich bei größerer Ausdehnung auch der Round Trip Delay erhöht. Möchte man nun
	eher kleine Nutzdaten zwischen Stationen schicken, so schickt man recht viele redundante Daten, da gepaddet werden muss und man belastet die Leitung dadurch unnötig, ganz
	abgesehen von der geringeren Nutzdatenrate durch das Aufblähen der Rahmen.\\
	Außerdem hat man bei großer Ausdehnung, falls eine hohe Last herrscht, eine \textit{geringe Effizienz}, da die Leitung dann oft "belegt ist" und die Kommunikation zwischen
	nahen Stationen, welche schon längst fertig sind, trotzdem weit weg gelegene Stationen lahm legen können.
	
\end{subexercise}
\end{exercise}


\end{document}
